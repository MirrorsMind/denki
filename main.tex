\documentclass[sigchi-a, authorversion]{acmart}
\usepackage{booktabs} % For formal tables
\usepackage{ccicons}  % For Creative Commons citation icons
\usepackage[utf8]{inputenc}
\settopmatter{printacmref=false}

% Copyright
\setcopyright{none}


% DOI
\acmDOI{00.000/00.0}

% ISBN
\acmISBN{123-4567-24-567/08/06}

%Conference
\acmConference[DENKWEISEN'18]{Denkweisen der Informatik}{Wintersemester 2018}{Wien, Österreich}
\acmYear{2018}
\copyrightyear{2018}


\begin{document}

\title{Beyond Robot and Frank}

\author{Michael Csida}
\affiliation{%
  \institution{TU Wien, 01528725}
  }
\email{krysanto@gmail.com}

\author{Georg Becker}
\affiliation{%
  \institution{TU Wien, 01228308}
  }
\email{GeorgBecker@gmx.at}

\author{Third Author}
\affiliation{%
  \institution{TU Wien, Matrikelnummer}
  }
\email{author@email}

\author{Fourth Author}
\affiliation{%
  \institution{TU Wien, Matrikelnummer}
  }
\email{author@email}

% The default list of authors is too long for headers.
\renewcommand{\shortauthors}{F. Author et al.}

\maketitle

\begin{abstract}
    Eine Zusammenfassung des Papers (150 Wörter)

\end{abstract}

% =============================================================================
\section{Introduction}
% =============================================================================

Gemeinsame Einleitung



% =============================================================================
\section{Denkweise Criminal Thinking}
% =============================================================================

Beschreiben sie generell, was diese Denkweise auszeichnet und welche Prinzipien ihr zu Grunde liegen.

Wie können diese Prinzipien auf die Probleme die in ``Robot and Frank'' thematisiert werde angewandt werden?

Wie könnten die Probleme die hier dargestellt werden aus dieser Denkweise heraus bearbeitet werden?

% =============================================================================
\section{Design Thinking(+ Creative Thinking)}
% =============================================================================

Beschreiben sie generell, was diese Denkweise auszeichnet und welche Prinzipien ihr zu Grunde liegen.

Wie können diese Prinzipien auf die Probleme die in ``Robot and Frank'' thematisiert werde angewandt werden?

Wie könnten die Probleme die hier dargestellt werden aus dieser Denkweise heraus bearbeitet werden?

Notes:

Welche Ziele werden verfolgt? Wie könnte das Framing für die Gestaltung des Roboters angelegt sein? 

Hunter möchte, dass sich der Roboter um Frank, Hunters Vater kümmert. (Motiv unbekannt, könnte sein, dass Hunter sich nicht selbst kümmern möchte oder kann)
Der Roboter ist Humanoid gestaltet umd die Interaktion mit ihm angenehmer zu gestalten.
Der Roboter ist designed eine Vertrauensbasis mit den Nutzer_innen aufzubauen.

Hat die stattfindende »Kreative Nutzung« auch positive Seiten?

Frank gewinnt ein Ziel in seinem Leben zurück

Wer macht was, wann, wo, warum, wie und mit wem? (vgl. Vorlesung)

Der Roboter weckt Frank und kontrolliert dessen Tagesablauf um Franks geistige und körperliche Gesundheit zu steigern.
Frank erklährt dem Roboter das Schlösserknacken und Sicherheitssysteme der Bibliothek während des Tages.
Der Roboter lügt zu Beginn zu Frank um diesen dazu zu bringen ihn zu befolgen.
Der Roboter unterstützt Frank bei der Planung eines Einbruchs weil dies seine geistige Gesundheit verbessert.

Welche der Probleme sind durch mangelndes Design-Denken bzw. unterbelichtete Human Centeredness
verursacht?

Welche impliziten Annahmen stecken in diesem System, und begrenzen wie eine »Box« die möglichen
Formen, die ein solches System annehmen kann? Wie können Sie »Outside the Box« ansetzen?

Der Roboter bedenkt nicht, dass der Patient bei Ausfall oder Defekt auf sich allein gestellt ist.
Das System nimmt an, dass es nicht
Das System nimmt an, dass die Nutzer_innen das System nicht missbrauchen wollen.
Das System nimmt an, dass die Nutzer_innen nicht zurechenungsfähig sind.
Das System nimmt an, dass wer das Passwort besitzt den Roboter Befehle erteilen darf.

Wie könnten Benutzer_innen in den Gestaltungsprozess eingebunden werden?

Inwiefern spiegeln die Produkte etwas Positives, Erstrebenswertes auf die Nutzer_innen zurück? Möchten
diese mit den Produkten gesehen werden?

Frank möchte nicht, dass die Bibliothekarin den Roboter sieht.

% =============================================================================
\section{Denkweise 3}
% =============================================================================

Beschreiben sie generell, was diese Denkweise auszeichnet und welche Prinzipien ihr zu Grunde liegen.

Wie können diese Prinzipien auf die Probleme die in ``Robot and Frank'' thematisiert werde angewandt werden?

Wie könnten die Probleme die hier dargestellt werden aus dieser Denkweise heraus bearbeitet werden?

% =============================================================================
\section{Denkweise 4}
% =============================================================================

Beschreiben sie generell, was diese Denkweise auszeichnet und welche Prinzipien ihr zu Grunde liegen.

Wie können diese Prinzipien auf die Probleme die in ``Robot and Frank'' thematisiert werde angewandt werden?

Wie könnten die Probleme die hier dargestellt werden aus dieser Denkweise heraus bearbeitet werden?


% =============================================================================
\section{Diskussion}
% =============================================================================

Welche Querverbindungen, Gemeinsamkeiten, Spannungsfelder, Widersprüche, Konflikte etc. sehen sie zwischen den unterschiedlichen Herangehensweisen?

Wie würden sie dieser Verbindungen bewerten oder priorisieren?

Wie könnten diese Denkweisen ineinander greifen um die Probleme in ``Robot and Frank'' zu lösen?

% =============================================================================
\section{Ausblick}
% =============================================================================

Wenn sie damit beauftragt wären die Technologien in ``Robot and Frank'' zu verbessern, wie würden sie das konkret angehen?

Welche Studien würden sie machen, was würden sie entwickeln, was würden sie wie testen?




\bibliography{Literature}
\bibliographystyle{ACM-Reference-Format}

\end{document}
