\documentclass[sigchi-a, authorversion]{acmart}
\usepackage{booktabs} % For formal tables
\usepackage{ccicons}  % For Creative Commons citation icons
\usepackage[utf8]{inputenc}
\settopmatter{printacmref=false}

% Copyright
\setcopyright{none}


% DOI
\acmDOI{00.000/00.0}

% ISBN
\acmISBN{123-4567-24-567/08/06}

%Conference
\acmConference[DENKWEISEN'18]{Denkweisen der Informatik}{Wintersemester 2018}{Wien, Österreich}
\acmYear{2018}
\copyrightyear{2018}


\begin{document}

\title{Beyond Robot and Frank}

\author{Michael Csida}
\affiliation{%
  \institution{TU Wien, 01528725}
  }
\email{krysanto@gmail.com}

\author{Second Author}
\affiliation{%
  \institution{TU Wien, Matrikelnummer}
  }
\email{author@email}

\author{Third Author}
\affiliation{%
  \institution{TU Wien, Matrikelnummer}
  }
\email{author@email}

\author{Fourth Author}
\affiliation{%
  \institution{TU Wien, Matrikelnummer}
  }
\email{author@email}

% The default list of authors is too long for headers.
\renewcommand{\shortauthors}{F. Author et al.}

\maketitle

\begin{abstract}
    Eine Zusammenfassung des Papers (150 Wörter)

\end{abstract}

% =============================================================================
\section{Introduction}
% =============================================================================

Gemeinsame Einleitung



% =============================================================================
\section{Denkweise Criminal Thinking}
% =============================================================================

Beschreiben sie generell, was diese Denkweise auszeichnet und welche Prinzipien ihr zu Grunde liegen.

Wie können diese Prinzipien auf die Probleme die in ``Robot and Frank'' thematisiert werde angewandt werden?

Wie könnten die Probleme die hier dargestellt werden aus dieser Denkweise heraus bearbeitet werden?

% =============================================================================
\section{Denkweise 2}
% =============================================================================

Beschreiben sie generell, was diese Denkweise auszeichnet und welche Prinzipien ihr zu Grunde liegen.

Wie können diese Prinzipien auf die Probleme die in ``Robot and Frank'' thematisiert werde angewandt werden?

Wie könnten die Probleme die hier dargestellt werden aus dieser Denkweise heraus bearbeitet werden?

% =============================================================================
\section{Denkweise 3}
% =============================================================================

Beschreiben sie generell, was diese Denkweise auszeichnet und welche Prinzipien ihr zu Grunde liegen.

Wie können diese Prinzipien auf die Probleme die in ``Robot and Frank'' thematisiert werde angewandt werden?

Wie könnten die Probleme die hier dargestellt werden aus dieser Denkweise heraus bearbeitet werden?

% =============================================================================
\section{Denkweise 4}
% =============================================================================

Beschreiben sie generell, was diese Denkweise auszeichnet und welche Prinzipien ihr zu Grunde liegen.

Wie können diese Prinzipien auf die Probleme die in ``Robot and Frank'' thematisiert werde angewandt werden?

Wie könnten die Probleme die hier dargestellt werden aus dieser Denkweise heraus bearbeitet werden?


% =============================================================================
\section{Diskussion}
% =============================================================================

Welche Querverbindungen, Gemeinsamkeiten, Spannungsfelder, Widersprüche, Konflikte etc. sehen sie zwischen den unterschiedlichen Herangehensweisen?

Wie würden sie dieser Verbindungen bewerten oder priorisieren?

Wie könnten diese Denkweisen ineinander greifen um die Probleme in ``Robot and Frank'' zu lösen?

% =============================================================================
\section{Ausblick}
% =============================================================================

Wenn sie damit beauftragt wären die Technologien in ``Robot and Frank'' zu verbessern, wie würden sie das konkret angehen?

Welche Studien würden sie machen, was würden sie entwickeln, was würden sie wie testen?




\bibliography{Literature}
\bibliographystyle{ACM-Reference-Format}

\end{document}
